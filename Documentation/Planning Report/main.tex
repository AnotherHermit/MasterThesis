\documentclass[a4paper, 12pt]{article}
\usepackage[utf8]{inputenc}

\usepackage[title,titletoc,toc]{appendix}

\usepackage[parfill]{parskip} % Space between paragraphs instead of indentation with this
%\usepackage{fullpage} % Less margins with this
%\usepackage[compact]{titlesec} % Less space around headings with this
%\usepackage{endfloat} % Put figures last with this

\usepackage{hyperref}
\usepackage{glossaries}
\usepackage[
    backend=biber,
    style=ieee,
    defernumbers=true]
    {biblatex}
\addbibresource{references.bib}


% Techniques
\newacronym{pm}{PM}{Photon Mapping}
\newacronym{vct}{VCT}{Voxel Cone Tracing}
\newacronym{ao}{AO}{Ambient Occlusion}
\newacronym{vpl}{VPL}{Virtual Point Light}
\newacronym{lpv}{LPV}{Light Propagation Volume}
\newacronym{ir}{IR}{Instant Radiosity}
\newacronym{rt}{RT}{Ray Tracing}

% Other things
\newacronym{gi}{GI}{Global Illumination}

% Libraries and such
\newacronym{ogl}{OpenGL}{Open Graphics Library}
\newacronym{ogles}{OpenGL ES}{Open Graphics Library for Embedded Systems}
\newacronym{aep}{AEP}{Android Extension Pack}
\newacronym{ocl}{OpenCL}{Open Computing Language}

% Don't know a good title
\newacronym{api}{API}{Application Programming Interface}
\newacronym{gpu}{GPU}{Graphics Processing Unit}
\newacronym{cpu}{CPU}{Central Processing Unit}
%

\title{Planning Report \\ \small{Version 0.2}}
\author{Conrad Wahlén \\ \texttt{conwa099@student.liu.se}}
\date{\today}

\begin{document}

\maketitle
\thispagestyle{empty}
\newpage


\section{Preliminary title}
\label{sec:Preliminary title}

The preliminary title for this master's thesis will be:

\textit{\acrlong{gi} using \acrlong{vct} on Mobile Devices}

\section{Background}
\label{sec:Background}

This master's thesis will be conducted on the behalf of Mindroad.

One of the most important parts of 3D rendering is the illumination since it grounds objects in the scene. Even though computing realistic lighting has been a goal of 3D rendering since the beginning of computer graphics there is still a lot of work to be done. Since the problem is so computationally heavy the really good looking solutions are not close to running in real time.
In the recent years however due to the increasing performance of hardware and some novel methods of approximating the scenes for lighting interactive methods for gi has been demonstrated. A very well written overview of the field 5 years ago can be found in~\cite{sotagi}.

The difficult part of illumination is the light bouncing and interacting with a lot of surfaces in the scene. So in order to model it, this behavior needs to be simulated or approximated in some way. If a really heavy approximation can be made the result could be something like AO, which only guesses where in the scene there would be less light.

A short demonstration of what bouncing light adds to the scene can be seen in the figures below. The bouncing light can be decently modeled with AO and an ambient light term for areas of the scene not in direct view of any light sources. However in this thesis a different model will be implemented which does not rely on such approximations.

Most techniques for gi does not run in real-time but recently a few of them have been demonstrated to run in real-time on high-end hardware. This thesis will investigate if the voxel cone tracing method scales down all the way to mobile level hardware.

Since the dawn of VR and AR is just upon us it would be of huge value if the devices that we put on our heads manage to produce impressive graphics without having to rely on server side graphics like described in~\cite{cloudlight}. While the idea of calculating graphics and lighting on a server and basically streaming the content to clients is very interesting, it does have a lot of limiting factors mainly when it comes to network connection.
Seeing where the limit is on this generations mobile devices is therefore still a relevant question, especially with Vulcan entering the picture.

\gls{gi} tries to model the effects of light in a 3D scene.

Illumination of a scene in computer graphics is a very computationally expensive task. In order to make a scene render in real-time a lot of development has been made to get effects that approximate certain effects without the expensive calculations.

For example there are many different variants of \gls{ao} to approximate indirect lighting effects and other techniques that pre-calculate advanced lighting effects for static objects. While many of these techniques make it possible to interact with scenes that would otherwise be static or stuttering, they either have visible errors or are only possible for static objects.

Recent developments in hardware along with novel ideas has made it possible to use methods previously only suitable for off-line rendering in real-time applications. For example \gls{pm}.

Since the hardware in mobile devices are making great progress (still far from high-end desktop solutions) and since it has already been proven once~\cite{gimobile} it would be interesting to see how far the latest mobile generation can push global illumination on limited hardware.

Global illumination tries to simulate correct lighting in a scene without using separate methods for certain effects. Rendering using \gls{pm} makes it possible to create direct and indirect light (both diffuse and specular), caustics and shadows making it one of the more popular techniques for off-line rendering when high quality is needed because it converges toward a correct solution when more photons are used.

The direct light is a scene is usually quite simple and can be done for quite a few light sources with different techniques such as deferred shading or lighting. Leading to a method using \glspl{vpl} where indirect light is approximated by creating many light sources.


\section{Problem Statement}
\label{sec:Problem Statement}

The problem statements for this thesis work will be:

\begin{itemize}
  \item \textit{How far can we take \gls{gi} on mobile devices using \gls{vct}?}
\end{itemize}

Previous results from~\cite{gimobile} show that \gls{gi} in real-time is possible on mobile devices using \glspl{vpl}. Even though the results are quite recent the pace of improvement in mobile computational power means that there should be room for more advanced approaches, like \gls{vct}.

\begin{itemize}
  \item \textit{What are the limiting factors of the mobile device? And are there any potential benefits on using mobile devices for \gls{gi}?}
\end{itemize}

The hardware structure on the mobile platforms are quite different from the desktop variant. Using a combined memory and very limited shared memory (if at all) between \gls{gpu} cores makes it necessary to adapt the code and the algorithms used.

\begin{itemize}
  \item \textit{Does \gls{vct} scale well enough to be used on limited hardware such as a mobile device?}
\end{itemize}

vct does have parameters that can be tuned for reaching higher performance. The number of cones traced is one which affects how much detail will be in the lights and shadows. Number of bounces makes a large difference on performance and the quality of the scene. Voxel resolution makes a difference for the amount of data that has to be searched when tracing and also the memory footprint of the algorithm overall.

\section{Research delimitations}
\label{sec:Research delimitations}

Thesis work is limited to 20 weeks, this means that there will be some delimitations described below in this section.

Android is the mobile platform of choice, since it supports both \gls{ogles}, \gls{ocl} and Vulcan. Android \acrshort{api} level 23 will be used since only the latest versions have support of the latest \gls{ogles} versions and Vulcan.

Even though it is very tempting to try to implement a solution using Vulcan to really make the most out of the limited hardware, it would probably take a larger share of the time than is justified. Instead development of the algorithm will be done using \gls{ogles} 3.1 + \gls{aep} or \gls{ogles} 3.2, depending on which is available. This should allow for faster development and more focus on the algorithm in question.

%TODO: Limitations in the method chosen, talk to Ingemar.

Since vct is quite complex and the focus is to make a optimized version running on a mobile device, there will be some delimitations.

The scenes rendered will contain no dynamic objects. The lights and the camera will be completely dynamic however. This means that the voxelization can be made either on the desktop or just once before the program starts. This probably depends largely on the performance of the voxelization implemented.

There will only be a single bounce of the light, this will be slighly compensated by an ambient term instead.

The scene will be very simple, only a cornell box with a few different setups to demonstrate some of the different behaviors of the light.


\section{Approach}
\label{sec:Approach}

%TODO: Describe method chosen and which order stuff will be made in.

\section{Time plan}
\label{sec:Time plan}

Project start: Week 6 \\
Half-time check: Week 16 \\
Final presentation: Week 33--34 (delayed because of summer break)

At the half-time check an implementation running on desktop should be completed.

The following sections of the report shall be completed:
\begin{itemize}
  \item Introduction
  \item Background
  \item Global Illumination
\end{itemize}

The following sections should have been started:
\begin{itemize}
  \item Framework:\@ \gls{ogles} on Android
  \item Voxel Cone Tracing
  \item Results
\end{itemize}

A detailed time plan can be found in appendix A\@.

\begin{refsection}
\nocite{*}
\printbibliography[heading=bibnumbered, title={Litterature Base}, subtype=litbase, prefixnumbers={LB}]
\end{refsection}

\printbibliography[heading=bibnumbered, notsubtype=litbase]

\newpage

\begin{appendices}

\section{Detailed time plan}
\label{app:timeplan}



\end{appendices}
\end{document}
